% Copyright 2013 Sean Kelleher. All rights reserved.
% Use of this source code is governed by a GPL
% license that can be found in the LICENSE file.

\documentclass{report}

\title{Investigation and Development of Self-Optimizing Data Structures}
\author{Sean Kelleher}

\begin{document}

\maketitle

\begin{abstract}

Traditional implementations of common data structures are generally designed to
be simple, being based on a single principle such as being organized in an array
or a linked list, and an appropriate implementation is chosen by an estimate of
the best fit for a problem domain. However, usage of a data structure can change
over the course of execution of a single program, or certain properties of the
data may change, causing the original implementation of the data structure to
become inadequate for dealing with its task at large. For instance, a list which
was originally being built up by a sequence of appending operations, taking
advantage of a list-based structure, may be subjected to a rally of random
accesses, which would achieve better performance if the list was instead being
stored in an array. A matrix which was once dense and whose methods traversed
the entire 2-dimensional array it was stored in may save large numbers of
computations, upon becoming sparse, by shifting to a grid representation based
on linked-lists. This project aims to research and develop a suite of what will
be referred to as self-optimizing data structures, which will strive to organize
their underlying representation so as to take advantage of known patterns of
behaviour or emerging data properties.

\end{abstract}

\end{document}
