% Copyright 2013 Sean Kelleher. All rights reserved.
% Use of this source code is governed by a GPL
% license that can be found in the LICENSE file.

\documentclass{report}

\title{Final Year Project Report}
\author{Sean Kelleher}

\bibliographystyle{alpha}

\begin{document}

\maketitle

\tableofcontents

\chapter{Introduction}

The motivation for the research of this material came to me when reading about
using circularly linked lists for representing matrices in \cite{Knuth:97}. I
liked the idea, as it made traversing entire matrices trivial when the matrix
was sparse - the only problem I found with the implementation was that, to use
it, you had to be sure that the matrix was going to be sparse for the duration
of its existence. However, this type of insight into the use of a data structure
can only be intuited, or at best derived from testing the run time of its
application under various inputs.  However, testing an application in this way
can lack the comprehensiveness of using real data.

Regardless of the above points, I believe the project has the ability to stand
on its own merit, even if just to present a suite of repeatable tests for
comparing the performance of different data structure representations. While
finding thresholds where a trade of representation is optimal and conducting
other such experiments, the facts and figures gleaned from these exercises
should prove illuminating about the structures being investigated.

\chapter{Approach}

\section{Analysis of data structure implementations}

1. Find the maximum "comfortable" size of each implementation.

We consider a "comfortable" size for a data structure implementation to be a
data structure size which takes less than a certain predefined amount of time to
construct.

\bibliography{report}

\end{document}
